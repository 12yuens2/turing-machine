\documentclass{article}

\newcommand{\n}[0]{\\[\baselineskip]}

\title{CS3052 Turing Machine Practical}
\date{2017-02-28}
\author{140011146}

\begin{document}

\pagenumbering{gobble}
\maketitle

\newpage
\pagenumbering{arabic}


\section{Introduction}
In this practical, we were tasked to implement a Turing machine that can read as input a description of a Turing machine and a text file and run that given TM description on the text file. My implementation is written in Haskell.
\n
We were also asked to design four Turing machines, two for the two given problems and two of our own problems. The descriptions for these machines can be found in the directory \texttt{TMs/}. My two other problems are reversing a string in place and searching for a substring within a string.
\n
Finally, we needed to run experiments on our TM descriptions on varied input to analyse the number of transitions the machine makes based on the length of the input.
\section{Turing machine implementation}

Documentation for my code is produced with \texttt{haddock} and can be found in the \texttt{docs/} directory.
\section{Turing machine descriptions}

\subsection{Palindrome}

\subsection{Binary addition}

\subsection{Reversing a string}

\subsection{Containing a substring}

\section{Experiments}


\end{document}